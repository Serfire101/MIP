% Metódy inžinierskej práce

\documentclass[10pt,twoside,slovak,a4paper]{article}

\usepackage[slovak, english]{babel}
%\usepackage[T1]{fontenc}
\usepackage[IL2]{fontenc} % lepšia sadzba písmena Ľ než v T1
\usepackage[utf8]{inputenc}
\usepackage{graphicx}
\usepackage{url} % príkaz \url na formátovanie URL
\usepackage{hyperref} % odkazy v texte budú aktívne (pri niektorých triedach dokumentov spôsobuje posun textu)

\usepackage{cite} 
% \addbibresource{sample.bib}

\pagestyle{headings}

\title{Describing the capabilities and cases of uses for sequence diagrams in software engineering \thanks{Semestrálny projekt v predmete Metódy inžinierskej práce, ak. rok 2021/22, vedenie: Shevchenko Serhii}} % meno a priezvisko vyučujúceho na cvičeniach

\author{Shevcheko Serhii\\[2pt]
	{\small Slovenská technická univerzita v Bratislave}\\
	{\small Fakulta informatiky a informačných technológií}\\
	{\small \texttt{xshevchenko@stuba.sk}}
	}

\date{\small 5. november 2021} 



\begin{document}

\maketitle

\begin{abstract}
The article will describe the general principles of constructing sequence diagrams, and provide optimal cases of their use in the field of software engineering modeling.

The main structural elements of a sequence diagram will be described, the rules for their use with corresponding examples.
\end{abstract}



\section{Introduction}

In the well-known graphical description language for modeling in the field of software development, UML, there are several types of diagrams, each of which has its own advantages and disadvantages.

Among the popular types of diagrams are class diagrams, component diagrams, and activity diagrams.

The problem with all the above diagrams is that they cannot simulate the dynamics of the model, the behavior and the relationship between "actors" over time. They represent the structure, but not the behavior of the model on the timeline.

To solve this problem, a sequence diagram was created. It, unlike other types of diagrams, shows the interactions of "actors" on the timeline, the exchange of "messages" (data) between them. And, again, this is all shown in projection on the time axis. That is, it allows you to explore the model in dynamics.

When to use a sequence diagram is detailed in the Part ~\ref{when}. Details about sequence diagram notation are in a~\ref{notation} Part. In a~\ref{building} Part there are tips about creating sequence diagram. And the ~\ref{Conclusion} Part is a Conclusion of this article.

\section{When to use a sequence diagram} \label{when}
A sequence diagram should be used primarily to visualize relationships between objects, taking into account the sequence of these very relationships.~\cite{IBM_SD}

This diagram is very useful for modeling synchronous services, it allows you to think through all the interactions between "actors" from the beginning to the end of the service life cycle on the timeline.
% Z obr.~\ref{f:rozhod} je všetko jasné. 

\begin{figure*}[tbh]
\centering
%\includegraphics[scale=1.0]{diagram.pdf}
% Aj text môže byť prezentovaný ako obrázok. Stane sa z neho označný plávajúci objekt. Po vytvorení diagramu zrušte znak \texttt{\%} pred príkazom \verb|\includegraphics| označte tento riadok ako komentár (tiež pomocou znaku \texttt{\%}).
% \caption{Rozhodujúci argument.}
% \label{f:rozhod}
\end{figure*}



\section{The notation} \label{notation}

% Základným problémom je teda\ldots{} Najprv sa pozrieme na nejaké vysvetlenie (časť~\ref{ina:nejake}), a potom na ešte nejaké (časť~\ref{ina:nejake}).\footnote{Niekedy môžete potrebovať aj poznámku pod čiarou.}

% Môže sa zdať, že problém vlastne nejestvuje\cite{Coplien:MPD}, ale bolo dokázané, že to tak nie je~\cite{Czarnecki:Staged, Czarnecki:Progress}. Napriek tomu, aj dnes na webe narazíme na všelijaké pochybné názory\cite{PLP-Framework}. Dôležité veci možno \emph{zdôrazniť kurzívou}.


% \subsection{Nejaké vysvetlenie} \label{ina:nejake}

Niekedy treba uviesť zoznam:

\begin{itemize}
\item jedna vec
\item druhá vec
	\begin{itemize}
	\item x
	\item y
	\end{itemize}
\end{itemize}

Ten istý zoznam, len číslovaný:

\begin{enumerate}
\item jedna vec
\item druhá vec
	\begin{enumerate}
	\item x
	\item y
	\end{enumerate}
\end{enumerate}


% \subsection{Ešte nejaké vysvetlenie} \label{ina:este}

\paragraph{Veľmi dôležitá poznámka.}
Niekedy je potrebné nadpisom označiť odsek. Text pokračuje hneď za nadpisom.



\section{Building a Sequence Diagram} \label{building}





\section{Conclusion} \label{Conclusion} % prípadne iný variant názvu






% týmto sa generuje zoznam literatúry z obsahu súboru literatura.bib podľa toho, na čo sa v článku odkazujete
\bibliography{literatura}
\bibliographystyle{plain} % prípadne alpha, abbrv alebo hociktorý iný
\end{document}
